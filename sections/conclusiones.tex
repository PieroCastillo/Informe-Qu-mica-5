\documentclass[../main.tex]{subfiles}
\begin{document}
\begin{itemize}
    \item 
    Después de llevar a cabo el experimento 1, se consiguió generar la representación gráfica 
    de la solubilidad del $KNO_3$, que se muestra en la "Figura número 04". 
    Los resultados obtenidos para las temperaturas de 54, 76, 85 y 91 
    grados Celsius fueron los siguientes: 100, 120, 152 y 201 g de $KNO_3$ 
    por cada 100 mL de $H_2O$, respectivamente. Al comparar estos resultados con los 
    valores teóricos, se observó que en todos los casos fueron mayores, 
    excepto para la temperatura de 91 °C, donde el valor obtenido fue mayor. 
    Por lo tanto, se logró cumplir con nuestro primer objetivo.
    \item En el experimento 2, se determinó que se requerían 50 mL de una solución de HCl 2M. 
    La concentración final de la solución de HCl preparada fue de 1M, 
    y el volumen total de la solución fue de 100 mL. De esta manera, se alcanzó nuestro objetivo para este experimento.
    \item En el experimento 3, se pudo determinar que se necesitaba una masa de 0.40 g de NaOH. 
    La concentración final de la solución de NaOH preparada fue de 0.1M, 
    y el volumen total de la solución fue de 100 mL. 
    Por lo tanto, se logró cumplir con nuestro tercer objetivo.
\end{itemize}
\end{document}