\documentclass[../main.tex]{subfiles}
\begin{document}
\subsection{Experimento 1}
En este experimento, los valores de solubilidad obtenidos fueron superiores a los esperados teóricamente.
Una posible razón de esto podría haber sido la falta de precisión del termómetro utilizado para medir las temperaturas.
El termómetro tenía una escala de 1 grados Celsius, lo que limitó la capacidad para obtener mediciones precisas.
Otro factor que pudo haber contribuido a la imprecisión de los valores obtenidos fue el tiempo y la temperatura a los 
que se calentaron los tubos de ensayo. Al calentar los tubos durante períodos prolongados a temperaturas cercanas al punto 
de ebullición, el agua contenida en ellos se evaporaba gradualmente. Dado que se trataba de un volumen pequeño de agua (1 mL), 
esto generó variaciones en los valores de solubilidad.
\end{document}