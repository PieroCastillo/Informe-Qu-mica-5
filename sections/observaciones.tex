\documentclass[../main.tex]{subfiles}

\usepackage{multirow}
\usepackage{float}
\usepackage{array}

\usepackage{gensymb}%degree symbol

\begin{document}

\begin{table}[H]
    \centering
    \begin{tabular}{c|c|c|c}
        \hline
        Tubos de Ensayo & Masa ($g$) & Volumen ($mL$) & Temperatura de Cristalización ($\degree C$) \\
        \hline
        #1    &  2.01      &   1    &  91  \\
        #2    &  1.52      &   1    &  85  \\
        #3    &  1.20      &   1    &  76  \\
        #4    &  1.00      &   1    &  54  \\
        \hline
    \end{tabular}
    \label{tab:exp1}
    \caption{Datos obtenidos del experimento 1}
\end{table}

\begin{table}[H]
    \centering
    \begin{tabular}{c|c|c}
        \hline
        Molaridad del $HCl_{(ac)}\:(M)$ & Volumen de $HCl_{(ac)}\:(mL)$ & Volumen usado de $H_2O$ puro $(mL)$\\
        \hline
        1 $M$ & 50 $mL$ & 50 $mL$ \\ 
        \hline
    \end{tabular}
    \label{tab:exp2}
    \caption{Datos obtenidos del experimento 2}
\end{table}

\begin{table}[H]
    \centering
    \begin{tabular}{c|c}
        \hline
        Masa de $NaCl \:(g)$ & Volumen de $H_2O$ usado $(mL)$\\
        \hline
        0.4     &   10 \\
        \hline
    \end{tabular}
    \label{tab:exp3}
    \caption{Datos obtenidos del experimento 3}
\end{table}

\end{document}