\documentclass[../main.tex]{subfiles}
\begin{document}
\subsection{Soluto}
Es el componente que se encuentra en menor proporción en una reacción química.\cite{petrucci}

\subsection{Solvente o disolvente}
Es el componente que se encuentra en mayor proporción en una reacción química.\cite{petrucci}

\subsection{Fiola de vidrio o matraz aforado}
Se utiliza generalmente para contener, almacenar y medir líquidos.

\subsection{Molaridad (M) o concentración molar}
Es el numero de moles de soluto por litro de solucion.\cite{chang}
\[M = \frac{n_{(sto)}}{V_{sol}} = \frac{w_{sto}}{\overline{M}\cdot V_{sol}}\]

\subsection{Porcentaje en peso}
Es una unidad comercial que se usa para intercambiar o comercializar solutos en una solución.\cite{chang}
\[\%masa\:de\:a=\frac{masa\:de\:a}{masa\:total}\cdot100\]

\subsection{Porcentaje en volumen}
Es una unidad comercial que se usa para intercambiar o comercializar solutos en una solución, 
aunque generalmente se utiliza cuando el soluto está en estado líquido.\cite{chang}
\[\%Volumen\:de\:a=\frac{volumen_a}{volumen\:total}\cdot100\]

\subsection{Partes por millón de masa}
Gramos de soluto por millón de gramos de solución.\cite{chang}
\[ppm = \frac{masa\:de\:soluto}{masa\:de\:soluci\acute{o}n}\cdot100\]

\subsection{Fracción molar de soluto}
Es la relación del número de moles de soluto y el número total de moles.
\[X = \frac{moles\,de\,soluto}{moles\,totales}\]

\subsection{Solubilidad de un soluto}
Es la cantidad máxima de un soluto que puede disolverse en una cantidad específica de disolvente.

\end{document}